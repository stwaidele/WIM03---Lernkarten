%
%		* ----------------------------------------------------------------------------
%		* "THE BEER-WARE LICENSE" (Revision 42/023):
%		* Ronny Bergmann <mail@rbergmann.info> wrote this file. As long as you retain
%		* this notice you can do whatever you want with this stuff. If we meet some day,
%		* and you think  this stuff is worth it, you can buy me a beer or a coffee in return. 
%		* ----------------------------------------------------------------------------
%
%		* ----------------------------------------------------------------------------
%		* "THE COFFEE-WARE LICENSE" (Revision 42/023):
%		* Ronny Bergmann <mail@rbergmann.info> actually only wrote the template for this file. 
%       * Stefan Waidele <Stefan@Waidele.info> wrote the content of the index-cards.
%       * As long as you retain
%		* this notice you can do whatever you want with this stuff. If we meet some day,
%		* and you think  this stuff is worth it, you can buy me a coffee in return. 
%		* ----------------------------------------------------------------------------
%
%
% Beispiel zur Dokumentvorlage für Din A6 Karteikarten 
% -- Version 1.8b --
%
\documentclass[a6paper,12pt,print,grid=none,parskip=half]{kartei}

\usepackage[utf8]{inputenc} %UTF8
\usepackage{amsfonts}   
\usepackage{amssymb}
\usepackage{amsmath}

\begin{document}
\newcommand{\limes}[2]{\lim_{#1 \rightarrow #2}}

%\elementarlimes{x->???}{Funktion}{Lösung}{Kurzform}{Bemerkungen}
\newcommand{\elementarlimes}[5]{
\begin{karte}{\[ \limes{x}{#1} #2 \]}
	\begin{huge}
	\[ \limes{x}{#1} #2 = #3 \]
	\end{huge}
	\begin{center}
	symbolische Kurzform: $"#4" = #3$
	\end{center}
	#5
	\end{karte}	
}

%\kurzfrage{Frage}{TextGroß}{TextKlein}
\newcommand{\kurzfrage}[3]{
\begin{karte}{\[#1\]}
	\begin{huge}\[#1 = #2\]\end{huge}
	\begin{center}#3\end{center}
	\end{karte}	
}

%	\begin{karte}[Thema]{Frage}[Bemerkung]
%	Antwort
%	\end{karte}

%	\begin{karte}{}[]
%	\end{karte}

\fach{Grenzwerte}
\kommentar{Definition}
\begin{karte}{Grenzwert einer Funktion $f$ für $x \rightarrow x_0 $ }
Wenn sich für $x \rightarrow x_0 $ die zugehörigen Funktionswerte 
einem Konstanten Wert $g (\in \mathbb{R})$ immer mehr nähern, 
egal, auf welche Weise $x$ gegen $x_0$ strebt, so sagt man,\\
\begin{center}
\textbf{$g$ ist der Grenzwert von $f(x)$ \\bei der Annäherung von $x$ gegen $x_0$;}\\
oder 
\[\lim_{x \rightarrow x_0} f(x) = g\]
"Limes von $f(x)$ für $x$ gegen $x_0$ gleich $g$" \\
\end{center}
oder
"$f(x)$ \textit{konvergiert} für $x \rightarrow x_0$ gegen den Grenzwert $g (\in \mathbb{R})$"\\

Wenn $f$ für  $x \rightarrow x_0 $ \textit{nicht konvergiert}, so sagt man:\\
"$f$ ist für $x \rightarrow x_0 $ \textit{divergent}."
\end{karte}

\begin{karte}{Links- und rechtsseitiger Grenzwert}
\textbf{Nähern wir uns "von Links"} (also mit größer werdenden Werten für $x$) der untersuchten Stelle $x_0$ an, 
so ermitteln wir den "linksseitigen Grenzwert" $\lim_{x \rightarrow x_0^-} f(x) = g$
 
\textbf{Nähern wir uns "von Rechts"} (also mit kleiner werdenden Werten für $x$) der untersuchten Stelle $x_0$ an, 
so ermitteln wir den "rechtsseitigen Grenzwert" $\lim_{x \rightarrow x_0^-} f(x) = g$

Wenn linksseitiger und rechtsseitiger Grenzwert übereinstimmen, spricht man von dem Grenzwert von $f$ in $x_0$
\[\mathop {\lim }\limits_{x \to {x_0}} f(x) = g \Leftrightarrow \mathop {\lim }\limits_{x \to x_0^ - } f(x) = \mathop {\lim }\limits_{x \to x_0^ + } f(x) = g
\]
\end{karte}

\begin{karte}{Grenzwert einer Funktion $f$ für $x \rightarrow \infty$}
Wenn für unbeschränkt wachsendes (oder analog: unbeschränkt fallendes) Argument $x$ $(x \rightarrow \infty$ die entsprechenden Funktionswerte $f(x)$ dem Zahlenwert $g (\in \mathbb{R})$ 
schließlich beliebig nahe kommen, so heißt die Funktion $f$ für $x \rightarrow \infty$ konvergent gegen den Grenzwert $g$.

\[\mathop {\lim }\limits_{x \to \infty } f(x) = g\]
"Limes von $f(x)$ für $x$ gegen Unendlich gleich $g$"

\[\mathop {\lim }\limits_{x \to -\infty } f(x) = g\]
"Limes von $f(x)$ für $x$ gegen Minus Unendlich gleich $g$"
\end{karte}


\fach{Grenzwerte}
\kommentar{Elementare Funktionen}

%\elementarlimes{x->???}{Funktion}{Lösung}{Kurzform}{Bemerkungen}
\elementarlimes{\infty}{x^n}{\infty}{\infty^n}{}
\elementarlimes{\infty}{\dfrac{1}{x^n}}{0}{\dfrac{1}{\infty}}{}
\elementarlimes{0}{x^n}{0}{0^n}{}
\elementarlimes{0^+}{\dfrac{1}{x^n}}{\infty}{\dfrac{1}{0^+}}{für $n > 0$;}
\elementarlimes{0^-}{\dfrac{1}{x^n}}{\left\lbrace\begin{array}{cl}+\infty, &\mbox{falls n gerade}\\-\infty, &\mbox{falls n ungerade}\end{array}\right.}{\dfrac{1}{0^-}}{$n \in N$}
\elementarlimes{\infty}{e^x}{\infty}{e^\infty}{}
\elementarlimes{-\infty}{e^x}{0^+}{e^{-\infty}}{}
\elementarlimes{-\infty}{e^{-x}}{\infty}{e^\infty}{}
\elementarlimes{\infty}{e^{-x}}{0^+}{e^{-\infty}}{}
\elementarlimes{0}{e^x}{1}{e^0}{(Spezialfall von $a^x$)}
\elementarlimes{0}{e^{-x}}{1}{e^0}{(Spezialfall von $a^{-x}$)}
\elementarlimes{\infty}{a^x}{\left\lbrace\begin{array}{cl}0 &\mbox{für $0<a<1$}\\1 &\mbox{für $a=1$}\\ \infty &\mbox{für $a>1$}\end{array}\right.}{a^\infty}{}
\elementarlimes{\infty}{a^{-x}}{\left\lbrace\begin{array}{cl}\infty &\mbox{für $0<a<1$}\\1 &\mbox{für $a=1$}\\ 0 &\mbox{für $a>1$}\end{array}\right.}{a^\infty}{}
\elementarlimes{\infty}{(\ln x)}{\infty}{\ln \infty}{}
\elementarlimes{1}{(\ln x)}{0}{\ln 1}{}
\elementarlimes{0^+}{(\ln x)}{-\infty}{\ln 0^+}{}

\fach{Grenzwerte}
\kommentar{Rechenregeln}
%\kurzfrage{Frage}{TextGroß}{TextKlein}
\kurzfrage{\lim c }{\lim c = c}{}
\kurzfrage{\lim (f \pm h)}{\lim f \pm \lim h}{$= g_1 \pm g_2$}
\kurzfrage{\lim (f \cdot h)}{\lim f \cdot \lim h}{$= g_1 \cdot g_2$}
\kurzfrage{\lim \dfrac{f}{h}}{\dfrac{\lim f}{\lim h}}{$= \dfrac{g_1}{g_2}$, sofern $h, g_2 \not= 0$;}
\kurzfrage{\lim f^n}{(\lim f)^n}{$= {g_1}^n$  ($n \in N$); }
\kurzfrage{\lim \sqrt[n]{f}}{\sqrt[n]{\lim f}}{$= \sqrt[n]{g_1}$ ($n \in N; f, g_1 \ge 0$);}
\kurzfrage{\lim e^f}{e^{\lim f}}{$= e^{g_1}$}
\kurzfrage{\lim \left(\ln f\right)}{\ln \left(\lim f\right)}{$= \ln g_1$, sofern $f, g_1 \ge 0$.}

\fach{Stetigkeit}
\kommentar{Definition}
%\begin{karte}{}
%\end{karte}
\begin{karte}{Was versteht man unter\\ "Stetigkeit einer Funktion"	?}
Eine Funktion ist (an einer Stelle $x_0$ oder in einem Intervall) stetig wenn folgende Voraussetzungen gegeben sind:
\begin{enumerate}
\item $f$ muss in $x_0$ definiert sein, das haißt $f(x)$ muss existieren
\item $f$ muss für $x \rightarrow x_0$ einen (endlichen) Grenzwert \\(und somit übereinstimmende rechts- und linksseitige Grenzwerte) besitzen.
\[\begin{array}{cll}
\limes{x}{x_0} f(x) &= \limes{x}{x_0^+} f(x) & \\ &= \limes{x}{x_0^-} f(x) \\ &= g & (\in R)
\end{array}
\]
\item der Grenzwert $g$ von $f$ für $x \rightarrow x_0$ muss nicht nur vorhanden sein, sondern er muss auch exakt mit dem Funktionswert an der Stelle $x_0$ übereinstimmen
\[\limes{x}{x_0} f(x) = f(x_0)
\]
\end{enumerate}
\end{karte}

\begin{karte}{Arten der Unstetigkeit}
Es gibt folgende klassische Unstetigkeiten:
\begin{enumerate}
\item \textbf{Sprung} \\z.B. bei Abschnittsweise definierten Funktionen. \\Stichwort "Sprungfixe Kosten"
\item \textbf{Pol}\\z.B. \[f(x)=\frac { 1 }{ x }\] 
\item \textbf{Lücke} \\z.B. an Stellen $x_0$ bei denen der Nenner 0 wird. \\Kann evt. geschlossen werden, in dem per Definition \[f(x_0) = \limes{x}{x_0} f(x)\] gesetzt wird.
\end{enumerate}
\end{karte}

\begin{karte}{Wann besitzt eine Funktion\\ einen (endlichen) Sprung?}
Die Funktion $f$ besitzt an der derelle $x_0$ \\
einen (endlichen) Sprung, wenn gilt:
\[g_{ 1 }\quad =\quad \underset { x\quad \rightarrow \quad { x }_{ 0 }^{ - } }{ \lim   } f(x)\quad \neq \quad \underset { x\quad \rightarrow \quad { x }_{ 0 }^{ + } }{ \lim   } f(x)\quad =\quad g_{ 2 }\]
\\
z.B: Sprungfixe Kosten, Portofunktion
\end{karte}

\begin{karte}{Wann besitzt eine Funktion \\einen Pol?}
Einer oder beide einseitigen Grenzwerte existiert nicht, \\das heißt $f$ strebt für $x\quad \rightarrow \quad { x }_{ 0 }$ gegen $\pm \infty$. \\Dann sagt man: \\
\\
$f$ hat an der Stelle $x_0$ eine Unendlichkeitsstelle oder einen Pol, \\
wenn $f$ für $x\rightarrow { x }_{ 0 }^{ + }$ und/oder $x\rightarrow { x }_{ 0 }^{ - }$ \\
den uneigentlichen Grenzwert $\infty$ oder $-\infty$ besitzt.\\
\\
z.B. Stückkosten bei Kostenfunktionen mit Grundgebühr (Stromkostenbeispiel)
\end{karte}

\begin{karte}{Wann besitzt eine Funktion\\ eine behebbare Unstetigkeitsstelle?}
$f$ hat an der Stelle $x_0$ eine behebbare Unstetigkeitsstelle, wenn gilt: \\
\[\lim _{ x\rightarrow { x }_{ 0 } }{ f(x) } =\lim _{ x\rightarrow { x }_{ 0 }^{ - } }{ f(x) } =\lim _{ x\rightarrow { x }_{ 0 }^{ + } }{ f(x) } =g \quad (\in \mathbb{R})\]
aber $g\neq  f\left( x_0 \right)$ bzw. $x_0\notin D_f$

\end{karte}

\begin{karte}{Frage}
Antwort
\end{karte}

\begin{karte}{Frage}
Antwort
\end{karte}



\begin{karte}{Frage}
Antwort
\end{karte}

\end{document}